\documentclass{article}

\title{A Survey of Parallel Programming Concepts}
\author{Kyle Clapper}

\begin{document}
\maketitle

\section{Introduction}
Brief explaination of what parallel programming is and motivate
why we need it.

\section{Parallel Programming Concepts}
This section will discuss the various parallel programming
concepts and paradigms.

\subsection{Concurrency vs Parallelism}
Short explaination of the difference.

\subsection{Shared State vs Message Passing}
Some languages rely on shared state need mechanisms like
locks and semaphores to manage it.

Other languages use no shared state and rely on message passing to
share data between parallel tasks.

\subsection{Task Parallelism vs Data Parallelism}
This is basically the difference between GPU parallelism
and CPU parallelism. SIMD vs running different tasks on
different CPUs.

\section{Concurrency in Modern Languages}
This section will be an overview of the various parallel programming
concepts modern day languages use. Each section will have an example of
a typical (toy) parallel program written in one of these languages. Each
section will talk about what types of parallel programming tasks
each language/system is and isn't good for.

\subsection{C, C++, and Java}
Shared state and traditional CPU parallelism problems.

\subsection{CUDA, OpenGL, and OpenMPI}
Modern day GPU programming

\subsection{Erlang}
Message passing and no shared state.

\subsection{Rust and Go}
Not sure, but I know both of these languages promise to make parallel computing
effortless. I think they both are a combination of C and Erlang style parallelism.

\subsection{Javascript}
Not parallel but concurrent (with the exception of web workers). Talk about
the event loop and maybe talk about web workers too.

\subsection{Racket}
Go over Racket's implementation of threads and the Concurrent ML concepts they employ.

\section{Current State of the Art}
Depending on time, this section would talk about some of the research underway on
new parallel programming paradigms, techniques, and technologies. Maybe mention
Sam Caldwell.

\end{document}
